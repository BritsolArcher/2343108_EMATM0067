\documentclass[a4paper,11pt]{article}

\usepackage{amsmath}
\usepackage{amssymb}
\usepackage{amsthm}
\usepackage{graphicx}
\usepackage{enumerate}
\usepackage{booktabs}
%you can add more packages using the same code above

%------------------

%\setlength{\topmargin}{0.0in}
%\setlength{\textheight}{10in}
%\setlength{\oddsidemargin}{0.0in}
%\setlength{\evensidemargin}{0.0in}
%\setlength{\textwidth}{6.5in}

%-------------------
\newtheorem{theorem}{Theorem}[section]
\newtheorem{proposition}[theorem]{Proposition}
\newtheorem{lemma}[theorem]{Lemma}
\newtheorem{corollary}[theorem]{Corollary}
\newtheorem{conjecture}[theorem]{Conjecture}


\theoremstyle{definition}
\newtheorem{definition}[theorem]{Definition}
\newtheorem*{example}{Example}

%------------------

%Everything before begin document is called the pre-amble and sets out how the document will look
%It is recommended you don't touch the pre-amble until you are familiar with LateX

\begin{document}
	
\title{Introduction to AI}
\date{}
\maketitle

\begin{abstract}
Some sorts of documents need abstracts. Others do not.
\end{abstract}

%The following code is not run because of the percentage sign, but you might find it useful for future work
% \tableofcontents

\section{Anlysis on the Penguin Dataset}

Start your document with words, written in full sentences and paragraphs.
%Using the percentage symbol, you can include comments in your code that do not appear in the output.
It is a good idea to break your document into sections and subsections
\subsection{EDA on Penguin Datasets}
This dataset comprimises features such as species, island and bill\_length\_mm. Firstly, I conducted an analysis 
to determine the type, number of values, number of unique values, and proportion of missing values for each feature within the dataset. The
anlysis is present in Table 1.

\begin{table}[h]
	\centering
	\caption{Summary of Features}
	\begin{tabular}{lcccc}
		\toprule
		Feature & Type & Count & Nunique & \% Null \\
		\midrule
		species & object & 344 & 3 & 0.000000 \\
		island & object & 344 & 3 & 0.000000 \\
		bill\_length\_mm & float64 & 342 & 164 & 0.581395 \\
		bill\_depth\_mm & float64 & 342 & 80 & 0.581395 \\
		flipper\_length\_mm & float64 & 342 & 55 & 0.581395 \\
		body\_mass\_g & float64 & 342 & 94 & 0.581395 \\
		sex & object & 333 & 2 & 3.197674 \\
		year & int64 & 344 & 3 & 0.000000 \\
		\bottomrule
	\end{tabular}
\end{table}

\noindent
Obviously, the proportion of missing values is not substantial. Therefore, rows containing missing values will be removed.
Subsequently, I employed Cramér's V statistic to measure the correlation between different features and generated a heatmap accordingly.
According to Figure 1, the features sex and year exhibit the quite low correlation with species; hence, these two features 
will not be applied to construct the prediction model.


\begin{figure}
    \centering
    \includegraphics[scale=0.45]{Cramér-s-V.png}
    \caption{Correlation between different features}
    \label{fig: }
\end{figure}




\subsection{Model Construction}

Lists can be numbered or ununmbered, and you can have sub-list inside a list.

\begin{enumerate}
	\item This is the first item in a numbered list.

	\item And the second
	
	\item 
	\begin{enumerate}
		\item Here the third item is in fact a numbered sub-list.
		\item item 2 of the numbered sub-list
	\end{enumerate}

	\item 
	\begin{itemize}
		\item Here the fourth item is an unnumbered sub-list.
		\item item 2 of the unnumbered sub-list
	\end{itemize}
\end{enumerate}

\subsection{Definitions and theorems}

Definitions, theorems, lemmas and so on, are 'enviroments' (like documents and lists). They need to begin and end.

\begin{definition}\label{my_def}
	A \emph{label} allows the user to tell Latex 'remember the numbering of that definition/theorem/equation'
\end{definition}

\begin{lemma} \label{my_lem}
	If something has a label, then we can refer to it, without knowing what number it is 
\end{lemma}

\begin{proof}
	For example, by calling up Definition \ref{my_def}. This works even if the ordering of things move.
	Note that the end of proof square box is already there
\end{proof}

\begin{theorem}
	And a final theorem
\end{theorem}

\begin{proof}
	Combining Definition \ref{my_def} with Lemman \ref{my_lem} we get Equation \ref{my_eqn} below.
\end{proof}

\section{Including maths}

Some maths, like $\varepsilon>0$ or $a_{23}=\alpha^3$, is written in-line. More important or complex maths is displayed on its own line.
For example, $$ \lim_{x\to\infty}f(x)=\frac{\pi}{4}.$$

Sometimes you need multiple lines of maths to line up nicely:

\begin{align*}
f(x+y)&=(x+y,-2(x+y))\\
&=(x,-2x)+(y,-2y)\\
&=f(x)+f(y),
\end{align*}

and sometimes you want to number lines in an equation

\begin{align}
A^{T} & =\begin{pmatrix}1 & 2\\
3 & 4
\end{pmatrix}^{T}\\
\label{my_eqn}  & =\begin{pmatrix}1 & 3\\
2 & 4
\end{pmatrix}
\end{align}

\section{References and Figures}
\LaTeX{} \cite{lamport94} also allows you to cite your sources. For more details on how this can be done, we refer the reader to \cite[sec:~Embedded System]{referencing}. But once you have a bibliography, you can use the cite command easily. Finally we add Figure \ref{fig:logo} to show how to add graphics. Note that we first need to make sure to have the graphic uploaded to Overleaf or saved in the same folder as your tex file (whichever is relevant to your case). Notice how the picture was resized using the scale command and that \LaTeX{} determine that the picture looks better above.

\begin{figure}
    \centering
    \includegraphics[scale=0.3]{logo-full-colour.png}
    \caption{The logo for the University of Bristol}
    \label{fig:logo}
\end{figure}


\begin{thebibliography}{99}

\bibitem{lamport94}
  Leslie Lamport,
  \textit{\LaTeX: a document preparation system},
  Addison Wesley, Massachusetts,
  2nd edition,
  1994.
  
\bibitem{referencing}
    Wikibooks,
    \textit{LaTeX/Bibliography Management},
    [0nline],
    Accessed at https://en.wikibooks.org/wiki/LaTeX/Bibliography\_Management,
    (DATE ACCESSED).
    

\end{thebibliography}

\end{document}